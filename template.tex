%-------------------------%
% Ultimate LaTeX template %
%-------------------------%

% You should install python and pygments (sudo easy_install pygments) and compile LaTeX
% using the -shell-escape argument in order to take advantage of syntax highlighting.

\documentclass{article}

%----------------------------%
% packages and configuration %
%----------------------------%

% LaTeX2e kernel bugfixes
\usepackage{fixltx2e}

% page size
% other options: a1paper, or for custom: paperwidth=8.5in, paperheight=11in
\usepackage[letterpaper]{geometry}

% margins: default margins are typographically optimal, but doesn't fit much text
% make 1in margins:
\usepackage{fullpage}
% make 1.5cm margins:
% \usepackage[cm]{fullpage}

% change default font to look more fancy
\renewcommand{\sfdefault}{bch}
\usepackage{inconsolata}
\renewcommand*{\familydefault}{\sfdefault}

% line spacing
\linespread{1.1}

% headers and footers
\usepackage{fancyhdr}

% determine last page
\usepackage{lastpage}

% calculate lengths in LaTeX
\usepackage{calc}

% Set up the header and footer
\pagestyle{fancy}
\fancyhf{} % sets both header and footer to nothing
\renewcommand{\headrulewidth}{0pt} % don't display header
\lfoot{\myname~(\myandrewid)} % Bottom left footer
\cfoot{\myhwname} % Bottom center footer
\rfoot{Page\ \thepage\ of\ \protect\pageref*{LastPage}} % Bottom right footer
\renewcommand\footrulewidth{0.4pt} % Size of the footer rule

% colors
\usepackage[usenames,dvipsnames,svgnames,table]{xcolor}
% section heading colors
\usepackage[compact]{titlesec}
\titleformat*{\section}{\normalfont\Large\bfseries\color{flair}}
\titleformat*{\subsection}{\normalfont\large\bfseries\color{flair}}
\titleformat*{\subsubsection}{\normalfont\normalsize\bfseries\color{flair}}
\titleformat*{\paragraph}{\normalfont\normalsize\bfseries\color{flair}}
\titleformat*{\subparagraph}{\normalfont\normalsize\bfseries\color{flair}}

% math
\usepackage{amsmath, amsfonts, amsthm, amssymb}
% fix amsmath quirks
\usepackage{mathtools}
% allow bold greek letters
\usepackage{bm}
% theorem environments
\usepackage{mdframed}
\theoremstyle{plain}

\mdfdefinestyle{quoted}{
hidealllines=true,
leftline=true,
innertopmargin=5pt,
innerbottommargin=5pt,
linewidth=4pt,
linecolor=gray!40,
backgroundcolor=gray!5,
innerrightmargin=5pt,
skipabove=\topsep,
skipbelow=\topsep}
\mdfdefinestyle{theorem}{style=quoted,linecolor=theoremline,backgroundcolor=theorembg}
\mdfdefinestyle{exercise}{style=quoted,linecolor=exerciseline,backgroundcolor=exercisebg}
\mdfdefinestyle{remark}{style=quoted,linecolor=remarkline,backgroundcolor=remarkbg}
\mdfdefinestyle{warning}{style=quoted,linecolor=warningline,backgroundcolor=warningbg}

\newmdtheoremenv[style=theorem]{theorem}{Theorem}[section]
\newmdtheoremenv[style=theorem]{lemma}[theorem]{Lemma}
\newmdtheoremenv[style=theorem]{corollary}[theorem]{Corollary}

\theoremstyle{definition}

\newmdtheoremenv[style=theorem]{definition}[theorem]{Definition}
\newmdtheoremenv[style=theorem]{example}[theorem]{Example}
\newmdtheoremenv[style=theorem]{conjecture}[theorem]{Conjecture}
\newmdtheoremenv[style=exercise]{exercise}{Exercise}[section]
\newmdtheoremenv[style=exercise]{problem}[exercise]{Problem}
\newmdtheoremenv[style=exercise]{question}[exercise]{Question}
\newmdtheoremenv[style=exercise]{task}[exercise]{Task}

\theoremstyle{remark}

\newtheorem*{remark}{Remark}
\surroundwithmdframed[style=remark]{remark}
\newtheorem*{warning}{Warning}
\surroundwithmdframed[style=warning]{warning}
\newtheorem*{solution}{Solution}

% code
% syntax highlighting
\usepackage{minted}
% change line number style
\renewcommand{\theFancyVerbLine}{
  \sffamily\textcolor[rgb]{0.5,0.5,0.5}{\scriptsize\arabic{FancyVerbLine}}}
% color scheme
\usemintedstyle{default}
% verbatim mode: \begin{verbatim} \end{verbatim}
\usepackage{verbatim}

% pseudocode
\usepackage{algorithmicx} % use /begin{algorithmic}[1] for line numbers
\usepackage{algpseudocode}
% line number fix
\algrenewcommand{\alglinenumber}[1]{\color[rgb]{0.5,0.5,0.5}\scriptsize#1}

% disable paragraph indents
\setlength\parindent{0pt}

% pdf links
% use a link with \hyperref[label_name]{link text}
% use a url with \url{url}
% use a href with \href{url}{Text}
\usepackage{hyperref}
\hypersetup{colorlinks=true,pdfborder={1 1 1 [3]},%
citebordercolor={.6 .6 .6},linkbordercolor={.6 .6 .6},%
citecolor=linkcolor,urlcolor=linkcolor,linkcolor=linkcolor}

% allow images
% use an image with \includegraphics{filename}
\usepackage{graphicx}

% tikz
\usepackage{tikz}

% override enumeration styles
% usage: \begin{enumerate}[(a)]
\usepackage{enumerate}

% fix typography
\usepackage{microtype}

%--------%
% macros %
%--------%

% homework header
\newcommand{\homeworkheader}{
  \makebox[\textwidth]{
    \colorbox{headerbg}{
      \vbox{
        \vspace{3mm}
        \hbox to \textwidth { {\bf \large \color{flair} \mycourse \hfill \myhwname} }
        \hbox to \textwidth { \myname~(\texttt{\myandrewid @andrew.cmu.edu}) \hfill \mydate }
        \ifcsname mynote\endcsname % include a note only if it is defined
        \hbox to \textwidth { \mynote \hfill }
        \fi
        \vspace{3mm}
      }
    }
  }
}

\renewcommand{\newlanguage}[1]{
% use mathescape to render LaTeX in code
\newminted{#1}{linenos,numbersep=5pt,bgcolor=codebg,tabsize=2}
\newmint[#1short]{#1}{bgcolor=codebg}}

% math shortcuts
\DeclareMathOperator*{\Var}{Var}
\newcommand{\E}{\mathbb{E}}
\newcommand{\naturals}{\mathbb{N}}
\newcommand{\reals}{\mathbb{R}}
\newcommand{\integers}{\mathbb{Z}}
\newcommand{\powerset}{\mathcal{P}}
\newcommand{\integral}[4]{\int_{#1}^{#2} #3 \,\mathrm{d}#4}
\newcommand{\derivative}[2]{\frac{\mathrm{d}#1}{\mathrm{d}#2}}

%--------------%
% theme colors %
%--------------%

% letter digits must be capitalized
\definecolor{flair}{HTML}{046380} % refers to the color of the titles/headings
\definecolor{headerbg}{HTML}{F9F8EC}
\definecolor{codebg}{HTML}{F9F8EC}
\definecolor{theoremline}{HTML}{99CFFF}
\definecolor{theorembg}{HTML}{E5F3FF}
\definecolor{exerciseline}{HTML}{99FF99}
\definecolor{exercisebg}{HTML}{E5FFE5}
\definecolor{remarkline}{HTML}{FFFF99}
\definecolor{remarkbg}{HTML}{FFFFCC}
\definecolor{warningline}{HTML}{FF9999}
\definecolor{warningbg}{HTML}{FFE5E5}
\definecolor{linkcolor}{HTML}{079CCA}

%----------------------%
% custom configuration %
%----------------------%

% define languages for syntax highlighting
% use \begin{langcode} \end{langcode} for block code and \langshort"something" for one liners
\newlanguage{sml}
\newlanguage{python}
\newlanguage{c}
\newlanguage{bash}
\newlanguage{latex}

% for use with \homeworkheader
\newcommand{\myname}{Jeffrey Zhang}
\newcommand{\myandrewid}{jczhang}
\newcommand{\mydate}{\today}
\newcommand{\mycourse}{15-359 A: Probability and Computing}
\newcommand{\myhwname}{\LaTeX{} Template}
% comment this out if you don't need a note in the homework header:
\newcommand{\mynote}{Collaborators: none}

%--------------%
% the document %
%--------------%

\begin{document}

% comment this out if you don't want a homework header
\homeworkheader

\section{Usage}
\subsection{Download}

Go to \url{https://github.com/jczhang/latex.git} to get the latest \LaTeX{} template.

\subsection{Installation}

This template requires the \texttt{minted} package, which uses Python's Pygments to provide syntax highlighting for code. Install \texttt{python}, then \texttt{pygments} with

\bashshort|sudo easy_install pygments|

When compiling your \LaTeX{} document, make sure to use the \texttt{-shell-escape} argument for \texttt{pdflatex} so that \LaTeX{} can run Python.

\begin{warning}
This enables \LaTeX{} packages to potentially run arbitrary code on your machine. Make sure you know what you're doing.
\end{warning}

If you use Sublime Text's \LaTeX Tools plugin, you can change this by going to \textbf{Preferences} $\blacktriangleright$ \textbf{Browse Packages...} and editing \texttt{LaTeX.sublime-build} under the \texttt{LaTeXTools} directory. If you're on Sublime Text 3, this directory might be compressed under \texttt{../Installed Packages/LaTeXTools.sublime-package} (this is actually a \texttt{.zip} file).

\section{Features}

\subsection{Components}

The \verb|\homeworkheader| command can be used to display the header at the top of this page. To use it, edit these commands:

\begin{latexcode}
\newcommand{\myname}{Jeffrey Zhang}
\newcommand{\myandrewid}{jczhang}
\newcommand{\mydate}{\today}
\newcommand{\mycourse}{15-359 A: Probability and Computing}
\newcommand{\myhwname}{\LaTeX{} Template}
% comment this out if you don't need a note in the homework header:
\newcommand{\mynote}{Collaborators: none}
\end{latexcode}

The footer is automatically populated with the corresponding values.

\subsection{Environments}

Some convenient environments have been included. They are \texttt{theorem}, \texttt{lemma}, \texttt{corollary}, \texttt{definition}, \texttt{example}, \texttt{conjecture}, \texttt{exercise}, \texttt{problem}, \texttt{question}, \texttt{task}, \texttt{remark}, \texttt{warning}, \texttt{solution}, \texttt{proof}.

\subsubsection{Theorem-like}

\begin{theorem}[Pythagorean Theorem]
In a right triangle with legs $a$, $b$ and hypotenuse $c$,
$$a^2 + b^2 = c^2.$$
\end{theorem}

\begin{lemma}[Pumping lemma for regular languages]
$$\begin{array}{l}
(\forall  L\subseteq \Sigma^*)  \\
\quad      (\mbox{regular}(L) \Rightarrow \\                            
\quad      ((\exists p\geq 1) ( (\forall w\in L) ((|w|\geq p) \Rightarrow \\
\quad\quad ((\exists x,y,z \in \Sigma^*) (w=xyz \land (|y|\geq 1 \land |xy|\leq p \land
(\forall i\geq 0)(xy^iz\in L))))))))
\end{array}$$
\end{lemma}

\begin{corollary}
The field of complex numbers is the algebraic closure of the field of real numbers.
\end{corollary}

\subsubsection{Explanation}

\begin{definition}
A Turing machine is a 7-tuple $M = (Q, \Gamma, b, \Sigma, \delta, q_0, F)$ where
\begin{itemize}
  \item $Q$ is a finite, non-empty set of states
  \item $\Gamma$ is a finite, non-empty set of the tape alphabet/symbols
  \item $b \in \Gamma$ is the blank symbol
  \item $\Sigma \subseteq \Gamma \setminus \{b\}$ is the set of input symbols
  \item $q_0 \in Q$ is the initial state
  \item $F \subseteq Q$ is the set of final or accepting states
  \item $\delta : Q \setminus F \times \Gamma \to Q \times \Gamma \times \{L, R\}$ is a partial function called the transition function, where $L$ is left shift and $R$ is right shift.
\end{itemize}
\end{definition}

\begin{example}
Is $(n+1)! \in O(n!)$?

No, we have $(n + 1)! = (n + 1) \cdot n!$, which, for all $n > c$, is larger than $cn!$.
\end{example}

\begin{conjecture}[Goldbach's conjecture]
Every even integer greater than 2 can be expressed as the sum of two primes.
\end{conjecture}

\subsubsection{Problem-like}

\begin{exercise}[Positive correlation]
Show that if $\Pr[A \mid B] > \Pr[A]$, then $\Pr[B \mid A] > \Pr[B]$.
\end{exercise}

\begin{problem}[Relationship]
It's complicated.
\end{problem}

\begin{question}
Why do you think this is?
\end{question}

\begin{task}
Write test cases for your code.
\end{task}

\subsubsection{Information}

\begin{remark}
I should have used lorem ipsum to construct this section.
\end{remark}

\begin{warning}
Your hair's on fire!
\end{warning}

\begin{solution}
Divide the balls into three groups of four balls each. Compare two of the groups; if they are the same weight, the odd ball is in the third group. If they are different weights, give up.
\end{solution}

\begin{proof}
\begin{align*}
\sum_{i=1}^{k+1}i & = \left(\sum_{i=1}^{k}i\right) +(k+1)\\ 
& = \frac{k(k+1)}{2}+k+1 & \text{[by induction hypothesis]}\\
& = \frac{k(k+1)+2(k+1)}{2}\\
& = \frac{(k+1)(k+2)}{2}\\
& = \frac{(k+1)((k+1)+1)}{2}.
\end{align*}
\qedhere
\end{proof}

\subsection{Code}

Code highlighting is syntax-dependent. To use a lexer supported by \texttt{pygments}, use the \verb|\newlanguage{lang}| command, where \texttt{lang} is the name of the lexer. You can then use \verb|\begin{langcode} .. \end{langcode}| to create a code block and \verb|\langshort".."| to create a one-liner.

\begin{pythoncode}
# returns the first natural number
def hello():
  return 0
\end{pythoncode}

\smlshort|foldl (fn (a, b) => a + b) 0 [1, 2, 3]|

\subsection{Pseudocode}

Pseudocode typesetting is available from the \texttt{algorithmicx} package.

\begin{algorithmic}[1]
\If {$i\geq maxval$}
    \State $i\gets 0$
\Else
    \If {$i+k\leq maxval$}
        \State $i\gets i+k$
    \EndIf
\EndIf
\end{algorithmic}

\subsection{Math shortcuts}

Some math shortcuts have been defined to speed up typesetting.

\begin{center}
\begin{tabular}{cc}
\hline
\textbf{\LaTeX} & \textbf{Result}\\
\hline
\verb|\Var(X)| & $\Var(X)$ \\
\verb|\E[X]| & $\E[X]$ \\
\verb|\naturals| & $\naturals$ \\
\verb|\reals^4| & $\reals^4$ \\
\verb|\integers^2| & $\integers^2$ \\
\verb|\powerset(\{0, 1\})| & $\powerset(\{0, 1\})$ \\
\verb|\displaystyle\integral{0}{+\infty}{xe^x}{x}| & $\displaystyle\integral{0}{+\infty}{xe^x}{x}$ \\
\verb|\integral{}{}{x}{x}| & $\integral{}{}{x}{x}$ \\
\verb|\derivative{y}{x}| & $\derivative{y}{x}$ \\
\verb|\derivative{}{x}[f(x)]| & $\derivative{}{x}[f(x)]$ \\
\hline
\end{tabular}
\end{center}

\subsection{Other}

Links using the \texttt{hyperref} package, images using the \texttt{graphicx} package, drawings using the \texttt{tikz} package.

\section{Customization}

Besides the homework header text, the colors can be customized easily. Of course, you can hack this template to implement your own desired features. Check out the source code!

\section{Bugs/TODO}

\begin{itemize}
  \item Put a little margin above and below code blocks.
  \item Implement functional pseudocode commands for \texttt{algorithmicx}.
  \item Add background colors to pseudocode.
  \item Real code and pseudocode line numbers don't line up.
  \item Some warning messages show up because characters are missing in \texttt{inconsolata} or something.
  \item Currently untested on platforms other than MacTeX.
\end{itemize}

\end{document}
